\documentclass{article}
\usepackage[utf8]{inputenc}

\title{Math bits}
\author{andersx@chem.wisc.edu}
\date{May 2015}

\usepackage{natbib}
\usepackage{graphicx}
\usepackage{fullpage}
\usepackage{amsmath}

%\newcommand{\beq}{\begin{equation}}
%\newcommand{\eeq}{\end{equation}}
\renewcommand{\thesection}{\arabic{section}.}
\numberwithin{equation}{section}

\renewcommand{\theequation}{\thesection\arabic{equation}}
\renewcommand{\thesubsection}{\thesection\arabic{subsection}}


\begin{document}

% \maketitle
\section{DFTB2/CPE}
The DFTB2 energy\cite{dftb3} is given by:
\begin{equation}
    E_\mathrm{dftb2} = \sum_i \sum_{\mu\nu} n_i C_{\mu i}  C_{\nu i} H^0_{\mu\nu} + \frac{1}{2} \sum_{ab} V^\mathrm{rep}_{ab} + \frac{1}{2} \sum_{ab} q_a q_b \gamma_{ab}% + \frac{1}{3} \sum_{ab} q_a^2 q_b \Gamma_{ab}
\end{equation}
The CPE energy\cite{cpekaminski} is given by:
\begin{equation}
    E_{\mathrm{cpe}}\left[\mathbf{q}, \mathbf{c}\right] = \mathbf{c}^T \cdot \mathbf{M} \cdot \mathbf{q} + \frac{1}{2} \mathbf{c}^T \cdot \mathbf{N} \cdot \mathbf{c}\label{eq:cpe_energy}
\end{equation}
The DFTB2/CPE energy is given by:
\begin{equation}
    E_\mathrm{{dftb2/cpe}} = E_\mathrm{{dftb2}}\left[ \rho \right] + E_\mathrm{{cpe}}\left[\mathbf{q}, \mathbf{c}\right]
\label{eq:dftb2cpe_energy}
\end{equation}
The set of coefficients of the CPE response density basis that variationally minimizes the total CPE energy in Eqn.~\ref{eq:cpe_energy}  is given by:
\begin{equation}
    \mathbf{c}= -\mathbf{N}^{-1} \cdot \mathbf{M} \cdot \mathbf{q}
\end{equation}
\subsection{Hamiltonian shift}
The Hamiltonian shift is calculated using the chain rule:

\begin{eqnarray}
    \Delta H_{\mu\nu} &=& \frac{\partial E_{\mathrm{cpe}}\left[\mathbf{q}, \mathbf{c}\right]}{\partial \rho_{\mu\nu}}\\
    &=& \sum_i \frac{\partial E_{\mathrm{cpe}}\left[\mathbf{q}, \mathbf{c}\right]}{\partial c_i} 
    \frac{\partial c_i}{\partial \rho_{\mu\nu}}
    + \sum_i \frac{\partial E_{\mathrm{cpe}}\left[\mathbf{q}, \mathbf{c}\right]}{\partial q_i} 
    \frac{\partial q_i}{\partial \rho_{\mu\nu}}.
\end{eqnarray}
The CPE energy has no parametric dependence on the density, but depends implicitly via the charges and coefficients.
The first term is zero due to the CPE energy being variationally minimized with respect to the coefficients.
The first part of the second term is in the CPE charge-independent case:
\begin{equation}
    \frac{\partial E_{\mathrm{cpe}}\left[\mathbf{q}, \mathbf{c}\right]}{\partial q_i} = [\mathbf{c^T}  \cdot \mathbf{M}]_i
%    \frac{\partial E_{\mathrm{cpe}}\left[\mathbf{q}, \mathbf{c}\right]}{\partial q_i} = 
%    \mathbf{c^T} \cdot \left( \frac{\mathrm{\partial}\mathbf{M}}{\mathrm{\partial}q_i}\right) \cdot \mathbf{q} 
%    + [\mathbf{c^T}  \cdot \mathbf{M}]_i + \frac{1}{2}\mathbf{c}^T \cdot \left( \frac{\mathrm{\partial}\mathbf{N}}{\mathrm{\partial}q_i}\right) \cdot \mathbf{c}. 
\end{equation}
In the CPE charge-dependent case:
\begin{equation}
    \frac{\partial E_{\mathrm{cpe}}\left[\mathbf{q}, \mathbf{c}\right]}{\partial q_i} = 
    \mathbf{c^T} \cdot \left( \frac{\mathrm{\partial}\mathbf{M}}{\mathrm{\partial}q_i}\right) \cdot \mathbf{q} 
    + [\mathbf{c^T}  \cdot \mathbf{M}]_i + \frac{1}{2}\mathbf{c}^T \cdot \left( \frac{\mathrm{\partial}\mathbf{N}}{\mathrm{\partial}q_i}\right) \cdot \mathbf{c}. 
\end{equation}
The second half of the second term is calculated as:
\begin{equation}
    \frac{\partial q_i}{\partial \rho_{\mu\nu}} = \frac{1}{2} S_{\mu\nu} \qquad \mathrm{(in~the~current~implementation)}
\end{equation}
where $\rho_{\mu\nu}$ is the DFTB density matrix element
\begin{equation}
    \rho_{\mu\nu} = \sum_i^\mathrm{occ} n_i c_{\mu,i} c_{\nu,i}.
\end{equation}




\subsection{Energy gradient}
The energy gradient is given by:

\begin{equation}
    \frac{\mathrm{d} E_\mathrm{dftb2/cpe}}{\mathrm{d} R_{kx}} = \frac{\mathrm{d} E_\mathrm{{dftb2}}\left[ \rho \right]}{\mathrm{d} R_{kx}} + \frac{\mathrm{d} E_\mathrm{{cpe}}\left[\mathbf{q}, \mathbf{c}\right]}{\mathrm{d} R_{kx}}
\label{eq:dftb2cpe_derivative}
\end{equation}
The additive DFTB2/CPE energy-gradient term is given by:
\begin{equation}
    \frac{\mathrm{d} E_\mathrm{{cpe}}\left[\mathbf{q}, \mathbf{c}\right]}{\mathrm{d} R_{kx}} 
    = \sum_i \frac{\mathrm{d} E_\mathrm{{cpe}}\left(\mathbf{q}, c_{i\mu}(R_{kx})\right)}{\mathrm{d} c_{i\mu}}    
        \frac{\mathrm{d} c_{i\mu}(R_{kx})}{\mathrm{d} R_{kx}} 
    + \sum_i \frac{\mathrm{d} E_\mathrm{{cpe}}\left(q_i(R_{kx}), \mathbf{c}\right)}{\mathrm{d} q_i}    
        \frac{\mathrm{d} q_i(R_{kx})}{\mathrm{d} R_{kx}} 
    + \frac{\partial E_\mathrm{{cpe}}\left[\mathbf{q}, \mathbf{c}\right]}{\partial R_{kx}}
\end{equation}
The first derivative term is zero, since the CPE energy is variationally optimized with respect to the $\mathbf{c}$ coefficients. Note that there is no dependence on $\tau(q)$ in DFTB2. %DFTB3 requires an extra term as explained later.
\subsection{Dependence on $q_i$}
The second term can be divided into two factors. The first factor can be calculated as described in the previous section (it is the same term as found in the Hamiltonian-shift.) 
E.g. in the CPE charge-independent case:
\begin{equation}
    \frac{\partial E_{\mathrm{cpe}}\left[\mathbf{q}, \mathbf{c}\right]}{\partial q_i} = [\mathbf{c^T}  \cdot \mathbf{M}]_i
%    \frac{\partial E_{\mathrm{cpe}}\left[\mathbf{q}, \mathbf{c}\right]}{\partial q_i} = 
%    \mathbf{c^T} \cdot \left( \frac{\mathrm{\partial}\mathbf{M}}{\mathrm{\partial}q_i}\right) \cdot \mathbf{q} 
%    + [\mathbf{c^T}  \cdot \mathbf{M}]_i + \frac{1}{2}\mathbf{c}^T \cdot \left( \frac{\mathrm{\partial}\mathbf{N}}{\mathrm{\partial}q_i}\right) \cdot \mathbf{c}. 
\end{equation}
And in the CPE charge-dependent case:
\begin{equation}
    \frac{\partial E_{\mathrm{cpe}}\left[\mathbf{q}, \mathbf{c}\right]}{\partial q_i} = 
    \mathbf{c^T} \cdot \left( \frac{\mathrm{\partial}\mathbf{M}}{\mathrm{\partial}q_i}\right) \cdot \mathbf{q} 
    + [\mathbf{c^T}  \cdot \mathbf{M}]_i + \frac{1}{2}\mathbf{c}^T \cdot \left( \frac{\mathrm{\partial}\mathbf{N}}{\mathrm{\partial}q_i}\right) \cdot \mathbf{c}. 
\end{equation}
The 2nd factor can be calculated by:
\begin{equation}
    \frac{\mathrm{d} q_a(R_{kx})}{\mathrm{d} R_{kx}} 
        % = \sum_i \sum_\mu \sum_\nu n_i c_{i\mu} c_{i\nu} \frac{\partial S_{\mu\nu}}{\partial R_{kx}}
    = \sum_i^\mathrm{occ} n_i \sum_{\mu \in a} \sum_b \sum_{\nu \in b}
    c_{\mu,i} c_{\nu,i}\frac{\partial S_{\mu\nu}}{\partial R_{kx}}
\end{equation}
This term must be calculated in the DFTB2 gradient code (missing up to now). 
\subsection{(Explicit) dependence on $R_{kx}$}

\begin{equation}
    \frac{\partial E_\mathrm{{cpe}}\left[\mathbf{q}, \mathbf{c}\right]}{\partial R_{kx}} = 
    \mathbf{c^T} \cdot \left( \frac{\mathrm{\partial}\mathbf{M}}{\mathrm{\partial}R_{kx}}\right) \cdot \mathbf{q} 
    + [\mathbf{c^T}  \cdot \mathbf{M}]_i + \frac{1}{2}\mathbf{c}^T \cdot \left( \frac{\mathrm{\partial}\mathbf{N}}{\mathrm{\partial}R_{kx}}\right) \cdot \mathbf{c}. 
\end{equation}







\appendix
\section{Appendix A}
\subsection{Mulliken charges}
The Mulliken charge is given by:
\begin{equation}
    q_a = \sum_i^\mathrm{occ} n_i \sum_{\mu \in a} \sum_b \sum_{\nu \in b}
    c_{\mu,i} c_{\nu,i} S_{\mu\nu}
\end{equation}
The charge fluctuation is then: 
\begin{equation}
    \Delta q_a = q_a - q_a^0
\end{equation}
\subsection{Density matrix elements}
The density matrix is defined in terms of the coefficient matrix:
\begin{equation}
    \rho_{\mu\nu} = \sum_i^\mathrm{occ} n_i c_{\mu,i} c_{\nu,i}
\end{equation}
\subsection{Mulliken charges dependence on density matrix}

\begin{equation}
    \frac{\partial q_i}{\partial \rho_{\mu\nu}} = \frac{1}{2} S_{\mu\nu}
\end{equation}
\subsection{Tangent matrices}
\subsubsection{First order matrix (geometry)}
The explicit tangent matrix elements is
\begin{equation}
    \frac{\mathrm{\partial}\mathbf{N_{ij}}}{\mathrm{\partial}R_{kx}} =
    \frac{\mathrm{\partial}}{\mathrm{\partial}R_{kx}}  \iint \frac{\phi_i^\mathrm{cpe}\left(\mathbf{r}\right)\phi_j^\mathrm{cpe}\left(\mathbf{r}\right)}{\left| \mathbf{r} - \mathbf{r'}\right|} \mathrm{d}\mathbf{r}\mathrm{d}\mathbf{r'}
\end{equation}
This is already implemented.
\subsubsection{First order matrix (charge)}
The explicit tangent matrix elements is:
\begin{equation}
    \frac{\mathrm{\partial}\mathbf{N_{ij}}}{\mathrm{\partial}q_i} =
    \frac{\mathrm{\partial}}{\mathrm{\partial}q_i}  \iint \frac{\phi_i^\mathrm{cpe}\left(\mathbf{r}\right)\phi_j^\mathrm{cpe}\left(\mathbf{r}\right)}{\left| \mathbf{r} - \mathbf{r'}\right|} \mathrm{d}\mathbf{r}\mathrm{d}\mathbf{r'}
\end{equation}
This is already implemented.

\subsubsection{Second order matrix (geometry)}
The explicit tangent matrix elements is
\begin{equation}
    \frac{\mathrm{\partial}\mathbf{M_{ij}}}{\mathrm{\partial}R_{kx}} =
    f(R_{ij})\frac{\mathrm{\partial}}{\mathrm{\partial}R_{kx}}  \iint \frac{\phi_i^\mathrm{cpe}\left(\mathbf{r}\right)\phi_j^\mathrm{dftb}\left(\mathbf{r}\right)}{\left| \mathbf{r} - \mathbf{r'}\right|} \mathrm{d}\mathbf{r}\mathrm{d}\mathbf{r'}
    + \frac{\mathrm{\partial}f(R_{ij})}{\mathrm{\partial}R_{kx}}  \iint \frac{\phi_i^\mathrm{cpe}\left(\mathbf{r}\right)\phi_j^\mathrm{dftb}\left(\mathbf{r}\right)}{\left| \mathbf{r} - \mathbf{r'}\right|} \mathrm{d}\mathbf{r}\mathrm{d}\mathbf{r'}
\end{equation}
This is already implemented.
\subsubsection{Second order matrix (charge)}
The explicit tangent matrix elements is (for DFTB2):
\begin{equation}
    \frac{\mathrm{\partial}\mathbf{N_{ij}}}{\mathrm{\partial}q_k} =
    f(R_{ij})\frac{\mathrm{\partial}}{\mathrm{\partial}q_k}  \iint \frac{\phi_i^\mathrm{cpe}\left(\mathbf{r}, q_i\right)\phi_j^\mathrm{dftb2}\left(\mathbf{r}\right)}{\left| \mathbf{r} - \mathbf{r'}\right|} \mathrm{d}\mathbf{r}\mathrm{d}\mathbf{r'}
\end{equation}
This is already implemented, where as for DFTB3, the DFTB basis function is also dependent on the charge, and the integral derivative is thus different:
\begin{equation}
    \frac{\mathrm{\partial}\mathbf{N_{ij}}}{\mathrm{\partial}q_k} =
    f(R_{ij})\frac{\mathrm{\partial}}{\mathrm{\partial}q_k}  \iint \frac{\phi_i^\mathrm{cpe}\left(\mathbf{r}, q_i\right)\phi_j^\mathrm{dftb3}\left(\mathbf{r}, q_j\right)}{\left| \mathbf{r} - \mathbf{r'}\right|} \mathrm{d}\mathbf{r}\mathrm{d}\mathbf{r'}
\end{equation}
Looks like this derivative is wrongly (or not at all) implemented.


\bibliographystyle{plain}
\bibliography{references}
\end{document}

