\documentclass{article}
\usepackage[utf8]{inputenc}

\title{Math bits}
\author{andersx@chem.wisc.edu}
\date{May 2015}

\usepackage{natbib}
\usepackage{graphicx}

\begin{document}

% \maketitle

\section{Introduction}
The DFTB3 Hamiltonan elements are given by\cite{cpekaminski}
\begin{equation}
    H_{\mu\nu} = H^0_{\mu\nu} + S_{\mu\nu} \sum_c q_c \left(
        \frac{1}{2} \left( \gamma_{ac} + \gamma_{bc} \right) +
        \frac{1}{3} \left( q_a \Gamma_{ac} + q_b \Gamma_{bc} \right) +
        \frac{q_c}{6} \left( \Gamma_{ca} + \Gamma_{cb} \right) 
    \right)
    \label{eq:hamiltonian}
\end{equation}
The variational implicit dependence of the response density coefficients, $\mathbf{c}$, which must be added to the DFTB3 Hamiltonian matrix (Eqn.~\ref{eq:hamiltonian}) is derived form the chain rule:
\begin{eqnarray}
    \Delta H_{\mu\nu} &=& \frac{\partial E_{\mathrm{cpe}}\left[\mathbf{q}, \mathbf{c}\right]}{\partial \rho_{\mu\nu}}\\
    &=& \sum_i \frac{\partial E_{\mathrm{cpe}}\left[\mathbf{q}, \mathbf{c}\right]}{\partial q_i} 
    \frac{\partial q_i}{\partial \rho_{\mu\nu}},
\end{eqnarray}
where
\begin{equation}
    E_{\mathrm{cpe}}\left[\mathbf{q}, \mathbf{c}\right] = \mathbf{c}^T \cdot \mathbf{M} \cdot \mathbf{q} + \frac{1}{2} \mathbf{c}^T \cdot \mathbf{N} \cdot \mathbf{c},
\end{equation}
so that
\begin{equation}
    \frac{\partial E_{\mathrm{cpe}}\left[\mathbf{q}, \mathbf{c}\right]}{\partial q_i} = 
    \mathbf{c^T} \cdot \left( \frac{\mathrm{\partial}\mathbf{M}}{\mathrm{\partial}q_i}\right) \cdot \mathbf{q} 
    + [\mathbf{c^T}  \cdot \mathbf{M}]_i + \frac{1}{2}\mathbf{c}^T \cdot \left( \frac{\mathrm{\partial}\mathbf{N}}{\mathrm{\partial}q_i}\right) \cdot \mathbf{c}. 
\end{equation}










% \begin{equation}
%     E_\mathrm{{dftb3/c}} = E_\mathrm{{dftb3}}\left[ \rho \right] + E_\mathrm{{cpe}}\left[\mathbf{q}, \mathbf{c}\right],
% \label{eq:dftb3c_energy}
% \end{equation}
% The set of coefficients of the CPE response density basis that variationally minimizes the total DFTB3/C energy in Eqn.~\ref{eq:dftb3c_energy}  is given by:
% \begin{equation}
%     \mathbf{c}= -\mathbf{N}^{-1} \cdot \mathbf{M} \cdot \mathbf{q}
% \end{equation}
% \begin{figure}[h!]
% \centering
% \includegraphics[scale=1.7]{universe.jpg}
% \caption{The Universe}
% \label{fig:univerise}
% \end{figure}

\bibliographystyle{plain}
\bibliography{references}
\end{document}

