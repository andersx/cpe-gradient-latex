\documentclass{article}
\usepackage[utf8]{inputenc}

\title{DFTB2/CPE and DFTB3/CPE gradients}
\author{andersx@chem.wisc.edu}

\usepackage{natbib}
\usepackage{graphicx}
\usepackage{fullpage}
\usepackage{amsmath}

%\newcommand{\beq}{\begin{equation}}
%\newcommand{\eeq}{\end{equation}}
\renewcommand{\thesection}{\arabic{section}.}
\numberwithin{equation}{section}

\renewcommand{\theequation}{\thesection\arabic{equation}}
\renewcommand{\thesubsection}{\thesection\arabic{subsection}}


\begin{document}
% \maketitle
\section{Notation}
Summation over atom centers is denoted by summation over indices $a, b, c,$ etc. Summation over AO-basis functions is denoted by summation over indices such as $\mu, \nu,$ etc. The notation "$\mu \in a$" means that the basis function $\mu$ is centered on the atom $a$.
\\\\In the following sections we let $q_a$ be the \textit{Mulliken population} (always a strictly positive quantity), defined as:
\begin{equation}
    q_a = \sum_i^\mathrm{occ} n_i \sum_a \sum_{\mu \in a} \sum_b \sum_{\nu \in b}
    C_{\mu i} C_{\nu i} S_{\mu\nu}
\end{equation}
\\\\The charge fluctuation is then: 
\begin{equation}
    \Delta q_a = q_a - q_a^0
\end{equation}


\section{DFTB2/CPE energy}


\subsection{CPE energy}
The CPE energy is given by:\cite{cpekaminski}
\begin{equation}
    E_{\mathrm{cpe}} = \mathbf{c}^T \cdot \mathbf{M} \cdot \mathbf{q} + \frac{1}{2} \mathbf{c}^T \cdot \mathbf{N} \cdot \mathbf{c},\label{eq:cpe_energy}
\end{equation}
where the first order CPE-CPE Coulomb interaction matrix elements are given by:
\begin{equation}
    M_{ij} = f(R_{ij})\iint \frac{\phi_i^\mathrm{cpe}\left(\mathbf{r}\right)\phi_j^\mathrm{dftb2}\left(\mathbf{r'}\right)}{\left| \mathbf{r} - \mathbf{r'}\right|} \mathrm{d}\mathbf{r}\mathrm{d}\mathbf{r'}
\end{equation}
and the second order CPE-DFTB2 Coulomb interaction matrix elements are given by:
\begin{equation}
    N_{ij} = \iint \frac{\phi_i^\mathrm{cpe}\left(\mathbf{r}\right)\phi_j^\mathrm{cpe}\left(\mathbf{r'}\right)}{\left| \mathbf{r} - \mathbf{r'}\right|} \mathrm{d}\mathbf{r}\mathrm{d}\mathbf{r'}
\end{equation}
The CPE basis functions depend on the Mulliken population, while the DFTB basis functions only have a charge dependence in DFTB3.
\\\\The set of coefficients of the CPE response density basis that variationally minimizes the total CPE energy in Eqn.~\ref{eq:cpe_energy} is given (analytically) by:
\begin{equation}
    \mathbf{c}= -\mathbf{N}^{-1} \cdot \mathbf{M} \cdot \mathbf{q}
\end{equation}
Using the above relation, the CPE energy can be recast into:
\begin{eqnarray}
    E_{\mathrm{cpe}}
    &=& \mathbf{c}^T \cdot \mathbf{M} \cdot \mathbf{q} + \frac{1}{2} \mathbf{c}^T \cdot \mathbf{N} \cdot \mathbf{c} \\
    &=& -(\mathbf{N}^{-1} \cdot \mathbf{M} \cdot \mathbf{q})^T \cdot \mathbf{M} \cdot \mathbf{q} + \frac{1}{2} + (\mathbf{N}^{-1} \cdot \mathbf{M} \cdot \mathbf{q})^T \cdot \mathbf{N} \cdot (\mathbf{N}^{-1} \cdot \mathbf{M} \cdot \mathbf{q})\\
    &=& -(\mathbf{N}^{-1} \cdot \mathbf{M} \cdot \mathbf{q})^T \cdot \mathbf{M} \cdot \mathbf{q} + \frac{1}{2} (\mathbf{N}^{-1} \cdot \mathbf{M} \cdot \mathbf{q})^T  \cdot \mathbf{M} \cdot \mathbf{q}\\
    &=& -\frac{1}{2} (\mathbf{N}^{-1} \cdot \mathbf{M} \cdot \mathbf{q})^T  \cdot \mathbf{M} \cdot \mathbf{q}
\end{eqnarray}

\subsection{DFTB2 energy}
The DFTB2 energy is given by:\cite{dftb2}
\begin{equation}
    E_\mathrm{dftb2} = \sum_i^\mathrm{occ} n_i  \sum_\mu \sum_\nu C_{\mu i}  C_{\nu i} H^0_{\mu\nu} + \frac{1}{2} \sum_{ab} \Delta q_a \Delta q_b \gamma_{ab}+ \frac{1}{2} \sum_{ab} V^\mathrm{rep}_{ab} % + \frac{1}{3} \sum_{ab} q_a^2 q_b \Gamma_{ab}
\end{equation}
The DFTB2 Hamiltonian matrix elements are given by:\cite{dftb2}
\begin{equation}
    H_{\mu\nu}^{\mathrm{(dftb2)}} = H^0_{\mu\nu} + \frac{1}{2}S_{\mu\nu} \sum_c \left( \gamma_{ac} + \gamma_{bc} \right)\Delta q_c 
\end{equation}
\subsection{Combined DFTB2/CPE energy}
The full DFTB2/CPE energy is given by:
\begin{eqnarray}
    E_\mathrm{{dftb2/cpe}} &=& \sum_i^\mathrm{occ} n_i \sum_{\mu} \sum_{\nu}  C_{\mu i} C_{\nu i} H_{\mu\nu}^{\mathrm{(dftb2)}}  
    + \frac{1}{2} \sum_{ab} V^\mathrm{rep}_{ab}
    + E_{\mathrm{cpe}} \nonumber\\
    &=& \underbrace{\sum_i^\mathrm{occ} n_i  \sum_\mu \sum_\nu C_{\mu i}  C_{\nu i} H^0_{\mu\nu}}_{E_\mathrm{H0}}
        + \underbrace{\frac{1}{2} \sum_{ab} \Delta q_a \Delta q_b \gamma_{ab}}_{E_\gamma}
    + \underbrace{\frac{1}{2} \sum_{ab} V^\mathrm{rep}_{ab}}_{E_\mathrm{rep}}
%        && \underbrace{- \frac{1}{2} \sum_i^\mathrm{occ} n_i \sum_{\mu} \sum_{\nu}  C_{\mu i} C_{\nu i} S_{\mu\nu} \left(
%    \frac{\partial E_{\mathrm{cpe}}\left[\mathbf{q}, \mathbf{c}\right]}{\partial q_a} +
%    \frac{\partial E_{\mathrm{cpe}}\left[\mathbf{q}, \mathbf{c}\right]}{\partial q_b}
%\right)}_{E_\mathrm{shift}}\nonumber\\
 + E_{\mathrm{cpe}}\\\nonumber\\
&=&  E_\mathrm{H0} + E_\gamma + E_\mathrm{rep} + E_\mathrm{cpe}\label{eq:shorthand}
\end{eqnarray}
The CPE Hamiltonian shift is given by:\cite{gieseyork2012}
\begin{equation}
    \Delta H_{\mu\nu}^{\mathrm{(cpe)}} =  \frac{1}{2} S_{\mu\nu} \left(
    \frac{\partial E_{\mathrm{cpe}}\left[\mathbf{q}, \mathbf{c}\right]}{\partial q_a} +
    \frac{\partial E_{\mathrm{cpe}}\left[\mathbf{q}, \mathbf{c}\right]}{\partial q_b}
\right) \qquad \mu \in a, \nu \in b
\end{equation}
Note that here $q_a$ and $q_b$ are the Mulliken populations. Giese and York (2012) give the derivative in terms of the Mulliken charge and differ by a sign. See Appendix A for details.
\\\\The occupied orbital energies is given in the terms of the (optimized) coefficients and the matrix elements mentioned previously:
\begin{eqnarray}
    \sum_i^\mathrm{occ} n_i \varepsilon_i 
    &=& \sum_i^\mathrm{occ} n_i \sum_{\mu} \sum_{\nu}  C_{\mu i} C_{\nu i} H_{\mu\nu}\nonumber\\
    &=& \sum_i^\mathrm{occ} n_i \sum_{\mu} \sum_{\nu}  C_{\mu i} C_{\nu i} \left(H_{\mu\nu}^{\mathrm{(dftb2)}} + \Delta H_{\mu\nu}^{\mathrm{(cpe)}} \right)\nonumber\\
    &=& \sum_i^\mathrm{occ} n_i \sum_{\mu} \sum_{\nu}  C_{\mu i} C_{\nu i} H_{\mu\nu}^0
    + \frac{1}{2} \sum_i^\mathrm{occ} n_i \sum_a \sum_{\mu \in a} \sum_b \sum_{\nu \in b}  C_{\mu i} C_{\nu i} S_{\mu\nu} \sum_c \left( \gamma_{ac} + \gamma_{bc} \right)\Delta q_c \nonumber\\
    && +\ \frac{1}{2} \sum_i^\mathrm{occ} n_i  \sum_a \sum_{\mu \in a} \sum_b \sum_{\nu \in b} C_{\mu i} C_{\nu i} S_{\mu\nu} \left(
    \frac{\partial E_{\mathrm{cpe}}\left[\mathbf{q}, \mathbf{c}\right]}{\partial q_a} +
    \frac{\partial E_{\mathrm{cpe}}\left[\mathbf{q}, \mathbf{c}\right]}{\partial q_b} \right)\label{eq:orbital_energies}
\end{eqnarray}
Using the relation above, the energy can be calculated in terms of the orbital energies (as implemented in CHARMM), by isolating $E_\mathrm{H0}$ in Eqn.~\ref{eq:orbital_energies} and inserting into Eqn.~\ref{eq:shorthand}.

\begin{eqnarray}
    E_\mathrm{{dftb2/cpe}}
    &=& \sum_i^\mathrm{occ} n_i \varepsilon_i 
    - \frac{1}{2} \sum_i^\mathrm{occ} n_i \sum_a \sum_{\mu \in a} \sum_b \sum_{\nu \in b}  C_{\mu i} C_{\nu i} S_{\mu\nu} \sum_c \left( \gamma_{ac} + \gamma_{bc} \right)\Delta q_c \nonumber\\
    && \underbrace{- \ \frac{1}{2} \sum_i^\mathrm{occ} n_i \sum_a \sum_{\mu \in a} \sum_b \sum_{\nu \in b}  C_{\mu i} C_{\nu i} S_{\mu\nu} \left(
    \frac{\partial E_{\mathrm{cpe}}\left[\mathbf{q}, \mathbf{c}\right]}{\partial q_a} +
    \frac{\partial E_{\mathrm{cpe}}\left[\mathbf{q}, \mathbf{c}\right]}{\partial q_b} \right)}_{E_\mathrm{shift}} \nonumber\\
    && +\ \frac{1}{2} \sum_{ab} \Delta q_a \Delta q_b \gamma_{ab}
    + \frac{1}{2} \sum_{ab} V^\mathrm{rep}_{ab} + E_{\mathrm{cpe}}
\end{eqnarray}
The $E_\mathrm{shift}$ must be subtracted from the electronic energy to compensate for double counting when adding $E_\mathrm{cpe}$ to the electronic energy in terms of the occupied orbital energies.
\\\\The following relation is useful:
\begin{eqnarray}
    E_\mathrm{shift}
    &=& - \ \frac{1}{2} \sum_i^\mathrm{occ} n_i \sum_a \sum_{\mu \in a} \sum_b \sum_{\nu \in b}  C_{\mu i} C_{\nu i} S_{\mu\nu} \left(
    \frac{\partial E_{\mathrm{cpe}}\left[\mathbf{q}, \mathbf{c}\right]}{\partial q_a} +
    \frac{\partial E_{\mathrm{cpe}}\left[\mathbf{q}, \mathbf{c}\right]}{\partial q_b} \right) \nonumber\\
    &=& - \ \frac{1}{2} \sum_i^\mathrm{occ} n_i \sum_a \sum_{\mu \in a} \sum_b \sum_{\nu \in b}  C_{\mu i} C_{\nu i} S_{\mu\nu}
    \frac{\partial E_{\mathrm{cpe}}\left[\mathbf{q}, \mathbf{c}\right]}{\partial q_a} \nonumber\\
    && - \ \frac{1}{2} \sum_i^\mathrm{occ} n_i \sum_a \sum_{\mu \in a} \sum_b \sum_{\nu \in b}  C_{\mu i} C_{\nu i} S_{\mu\nu}
    \frac{\partial E_{\mathrm{cpe}}\left[\mathbf{q}, \mathbf{c}\right]}{\partial q_b} \nonumber\\
    &=& - \sum_i^\mathrm{occ} n_i \sum_a \sum_{\mu \in a} \sum_b \sum_{\nu \in b}  C_{\mu i} C_{\nu i} S_{\mu\nu}
    \frac{\partial E_{\mathrm{cpe}}\left[\mathbf{q}, \mathbf{c}\right]}{\partial q_a} \nonumber\\
    &=& -  \sum_a  \frac{\partial E_{\mathrm{cpe}}\left[\mathbf{q}, \mathbf{c}\right]}{\partial q_a} q_a
\end{eqnarray}
Using the above relation, the DFTB2/CPE energy can be simplified to:
\begin{equation}
    E_\mathrm{{dftb2/cpe}} = \sum_i^\mathrm{occ} n_i \varepsilon_i - \frac{1}{2}\sum_{ab} \left(q_a + q_a^0 \right)\Delta q_b \gamma_{ab}
    - \sum_a  \frac{\partial E_{\mathrm{cpe}}\left[\mathbf{q}, \mathbf{c}\right]}{\partial q_a} q_a + E_{\mathrm{cpe}}
\end{equation}


\clearpage
\section{DFTB2/CPE energy gradient}
The energy gradient is the derivative of the energy with respect to the nuclear coordinates under the following constraint:\cite{dftb3}
\begin{equation}
    - F_{kx} = \frac{\partial}{\partial R_{kx}} \left[ E_\mathrm{{dftb2/cpe}} - \sum_i^\mathrm{occ} n_i \varepsilon_i \left(\sum_{\mu} \sum_{\nu} C_{\mu i} C_{\nu i} S_{\mu\nu} - 1  \right) \right]
\end{equation}
Using the notation of Eq.~\ref{eq:shorthand}, we can rewrite this as:
\begin{equation}
    - F_{kx} = \frac{\partial}{\partial R_{kx}} \left[  E_\mathrm{H0} + E_\gamma + E_\mathrm{rep} + E_\mathrm{cpe} - \sum_i^\mathrm{occ} n_i \varepsilon_i \left(\sum_{\mu} \sum_{\nu} C_{\mu i} C_{\nu i} S_{\mu\nu} - 1 \right) \right]
\end{equation}
Separating the terms that appear in the standard DFTB2 gradient, and the DFTB2/CPE gradient, we arrive at the CPE gradient correction:
\begin{eqnarray}
    - F_{kx} &=& \frac{\partial}{\partial R_{kx}} \left[  E_\mathrm{H0} + E_\gamma + E_\mathrm{rep}  + E_\mathrm{cpe} - \sum_i^\mathrm{occ} n_i \varepsilon_i \left(\sum_{\mu} \sum_{\nu} C_{\mu i} C_{\nu i} S_{\mu\nu} - 1 \right) \right]\nonumber\\
             &=& \underbrace{\frac{\partial}{\partial R_{kx}} \left[  E_\mathrm{H0} + E_\gamma  + E_\mathrm{rep}- \sum_i^\mathrm{occ} n_i \varepsilon_i \left(\sum_{\mu} \sum_{\nu} C_{\mu i} C_{\nu i} S_{\mu\nu} - 1 \right) \right]}_{\mathrm{Same~as~the~DFTB2~gradient}}
    +\frac{\partial}{\partial R_{kx}} E_\mathrm{cpe} \nonumber\\
    &=& -F_{kx}^\mathrm{(dftb2)} + \frac{\partial E_\mathrm{cpe}}{\partial R_{kx}}
\end{eqnarray}
The last term is new, and is presented in the next sections.

\subsection{CPE gradient: $-F_{kx}^\mathrm{(cpe)} = \frac{\partial E_\mathrm{cpe}}{\partial R_{kx}}$}
The CPE energy depends explicitly on the coordinates via the Coulomb integrals, and implicitly on the coordinates via the CPE coefficients and the Mulliken population:

\begin{eqnarray}
    -F_{kx}^\mathrm{(cpe)} &=& \frac{\partial E_\mathrm{{cpe}}\left(R_{kx}\right)}{\partial R_{kx}} \nonumber\\
%-F_{kx}^\mathrm{(cpe)} &=& \frac{\partial E_\mathrm{{cpe}}\left[\mathbf{q}, \mathbf{c}\right]}{\partial R_{kx}} \nonumber\\
    &=& \sum_i \frac{\partial E_\mathrm{{cpe}}\left(\mathbf{q}, c_{i}(R_{kx})\right)}{\partial c_{i}}    
    \frac{\partial c_{i}(R_{kx})}{\partial R_{kx}} 
    + \sum_a \frac{\partial E_\mathrm{{cpe}}\left(q_a(R_{kx}), \mathbf{c}\right)}{\partial q_a}    
    \frac{\partial q_a(R_{kx})}{\partial R_{kx}} 
    + \frac{\partial E_\mathrm{{cpe}}\left[\mathbf{q}, \mathbf{c}\right]}{\partial R_{kx}}
\end{eqnarray}
The first derivative term is zero, since the CPE energy is variationally optimized with respect to the $\mathbf{c}$ coefficients. 
\subsubsection{Dependence on $q_a$}
The second term can be divided into two factors. The first factor can be calculated as described in the previous section (it is the same term as found in the Hamiltonian-shift.) - i.e.~the derivatives with respect to the Mulliken populations. 
In the CPE charge-independent case:
\begin{equation}
    \frac{\partial E_{\mathrm{cpe}}\left[\mathbf{q}, \mathbf{c}\right]}{\partial q_a} = [\mathbf{c^T}  \cdot \mathbf{M}]_a
\end{equation}
And in the CPE charge-dependent case:
\begin{equation}
    \frac{\partial E_{\mathrm{cpe}}\left[\mathbf{q}, \mathbf{c}\right]}{\partial q_a} = 
    \mathbf{c^T} \cdot \left( \frac{\mathrm{\partial}\mathbf{M}}{\mathrm{\partial}q_a}\right) \cdot \mathbf{q} 
    + [\mathbf{c^T}  \cdot \mathbf{M}]_a + \frac{1}{2}\mathbf{c}^T \cdot \left( \frac{\mathrm{\partial}\mathbf{N}}{\mathrm{\partial}q_a}\right) \cdot \mathbf{c}. 
\end{equation}
The second factor can be calculated for $a \neq k$ by:\cite{dftb3}
\begin{equation}
    \frac{\partial q_{a\neq k}}{\partial R_{kx}} 
    = \sum_i^\mathrm{occ} n_i \sum_{\mu \in a} \sum_{\nu \in k}  C_{\mu i} C_{\nu i}\frac{\partial S_{\mu\nu}}{\partial R_{kx}}
\end{equation}
or for $a=k$:
\begin{equation}
    \frac{\partial q_{k}}{\partial R_{kx}} 
    = \sum_i^\mathrm{occ} n_i \sum_{\mu \in k} \sum_{\nu \not\in k}  C_{\mu i} C_{\nu i}\frac{\partial S_{\mu\nu}}{\partial R_{kx}}
\end{equation}
This term must be calculated in the DFTB2 gradient code, where the derivative of the overlap matrix is already being calculated.

\subsubsection{(Explicit) dependence on $R_{kx}$}
The derivative with respect to $R_{kx}$ is written via the matrix derivatives.
\begin{equation}
    \frac{\partial E_\mathrm{{cpe}}\left[\mathbf{q}, \mathbf{c}\right]}{\partial R_{kx}} = 
    \mathbf{c^T} \cdot \left( \frac{\mathrm{\partial}\mathbf{M}}{\mathrm{\partial}R_{kx}}\right) \cdot \mathbf{q} 
    + \frac{1}{2}\mathbf{c}^T \cdot \left( \frac{\mathrm{\partial}\mathbf{N}}{\mathrm{\partial}R_{kx}}\right) \cdot \mathbf{c}. 
\end{equation}


\bibliographystyle{unsrt}
\bibliography{references}
\clearpage
\section*{Appendices}
\appendix
\section{Hamiltonian shift}

\subsection{Mulliken charges}
The Mulliken population is given by:
\begin{equation}
    q_a = \sum_i^\mathrm{occ} n_i \sum_{\mu \in a} \sum_{\nu}
    C_{\mu i} C_{\nu i} S_{\mu\nu}
\end{equation}
The charge fluctuation is then: 
\begin{equation}
    \Delta q_a = q_a - q_a^0
\end{equation}
Note that the \textit{Mulliken population} is always a positive number, and the \textit{Mulliken charge} is the negative of the Mulliken population.
\subsection{Density matrix elements}
The density matrix is defined in terms of the coefficient matrix:
\begin{equation}
    \rho_{\mu\nu} = \sum_i^\mathrm{occ} n_i C_{\mu i} C_{\nu i}
\end{equation}

\subsection{Hamiltonian shift}
The Hamiltonian shift is calculated using the chain rule:

\begin{eqnarray}
    \Delta H_{\mu\nu}^{\mathrm{(cpe)}} &=& \frac{\partial E_{\mathrm{cpe}}\left[\mathbf{q}, \mathbf{c}\right]}{\partial \rho_{\mu\nu}}\\
    &=& \sum_i \frac{\partial E_{\mathrm{cpe}}\left[\mathbf{q}, \mathbf{c}\right]}{\partial c_i} 
    \frac{\partial c_i}{\partial \rho_{\mu\nu}}
    + \sum_a \frac{\partial E_{\mathrm{cpe}}\left[\mathbf{q}, \mathbf{c}\right]}{\partial q_a} 
    \frac{\partial q_a}{\partial \rho_{\mu\nu}}\\
    &=& \frac{1}{2} S_{\mu\nu} \left(
    \frac{\partial E_{\mathrm{cpe}}\left[\mathbf{q}, \mathbf{c}\right]}{\partial q_a} +
    \frac{\partial E_{\mathrm{cpe}}\left[\mathbf{q}, \mathbf{c}\right]}{\partial q_b}
\right) \qquad \mu \in a, \nu \in b
\end{eqnarray}
Note that the last derivative is taken with respect to the Mulliken population. In Giese and York (2012) the negative sign is adopted.
\\\\The CPE energy has no parametric dependence on the density, but depends implicitly via the charges and coefficients.
The first term is zero due to the CPE energy being variationally minimized with respect to the coefficients.
The first part of the second term is in the CPE charge-independent case:
\begin{equation}
    \frac{\partial E_{\mathrm{cpe}}\left[\mathbf{q}, \mathbf{c}\right]}{\partial q_a} = [\mathbf{c^T}  \cdot \mathbf{M}]_a
\end{equation}
In the CPE charge-dependent case:
\begin{equation}
    \frac{\partial E_{\mathrm{cpe}}\left[\mathbf{q}, \mathbf{c}\right]}{\partial q_a} = 
    \mathbf{c^T} \cdot \left( \frac{\mathrm{\partial}\mathbf{M}}{\mathrm{\partial}q_a}\right) \cdot \mathbf{q} 
    + [\mathbf{c^T}  \cdot \mathbf{M}]_a + \frac{1}{2}\mathbf{c}^T \cdot \left( \frac{\mathrm{\partial}\mathbf{N}}{\mathrm{\partial}q_a}\right) \cdot \mathbf{c}. 
\end{equation}
See Giese and York (2012) for details.\cite{gieseyork2012}

% \section{Derivative matrix elements}
% 
% \subsection{First order matrix (geometry)}
% The matrix elements are:
% \begin{equation}
%     \frac{\mathrm{\partial} N_{ij}}{\mathrm{\partial}R_{kx}} =
%     \frac{\mathrm{\partial}}{\mathrm{\partial}R_{kx}}  \iint \frac{\phi_i^\mathrm{cpe}\left(\mathbf{r}\right)\phi_j^\mathrm{cpe}\left(\mathbf{r'}\right)}{\left| \mathbf{r} - \mathbf{r'}\right|} \mathrm{d}\mathbf{r}\mathrm{d}\mathbf{r'}
% \end{equation}
% This is already implemented.
% 
% \subsection{First order matrix (charge)}
% The matrix elements are:
% \begin{equation}
% \frac{\mathrm{\partial} N_{ij}}{\mathrm{\partial}q_k} =
%     \frac{\mathrm{\partial}}{\mathrm{\partial}q_k}  \iint \frac{\phi_i^\mathrm{cpe}\left(\mathbf{r}\right)\phi_j^\mathrm{cpe}\left(\mathbf{r'}\right)}{\left| \mathbf{r} - \mathbf{r'}\right|} \mathrm{d}\mathbf{r}\mathrm{d}\mathbf{r'}
% \end{equation}
% This is already implemented.
% 
% \subsection{Second order matrix (geometry)}
% The matrix elements are:
% \begin{equation}
%     \frac{\mathrm{\partial} M_{ij}}{\mathrm{\partial}R_{kx}} =
%     f(R_{ij})\frac{\mathrm{\partial}}{\mathrm{\partial}R_{kx}}  \iint \frac{\phi_i^\mathrm{cpe}\left(\mathbf{r}\right)\phi_j^\mathrm{dftb}\left(\mathbf{r'}\right)}{\left| \mathbf{r} - \mathbf{r'}\right|} \mathrm{d}\mathbf{r}\mathrm{d}\mathbf{r'}
%     + \frac{\mathrm{\partial}f(R_{ij})}{\mathrm{\partial}R_{kx}}  \iint \frac{\phi_i^\mathrm{cpe}\left(\mathbf{r}\right)\phi_j^\mathrm{dftb}\left(\mathbf{r'}\right)}{\left| \mathbf{r} - \mathbf{r'}\right|} \mathrm{d}\mathbf{r}\mathrm{d}\mathbf{r'}
% \end{equation}
% This is already implemented.
% 
% \subsection{Second order matrix (charge)}
% The matrix elements are (for DFTB2):
% \begin{equation}
% \frac{\mathrm{\partial} M_{ij}}{\mathrm{\partial}q_k} =
%     f(R_{ij})\frac{\mathrm{\partial}}{\mathrm{\partial}q_k}  \iint \frac{\phi_i^\mathrm{cpe}\left(\mathbf{r}, q_i\right)\phi_j^\mathrm{dftb2}\left(\mathbf{r'}\right)}{\left| \mathbf{r} - \mathbf{r'}\right|} \mathrm{d}\mathbf{r}\mathrm{d}\mathbf{r'}
% \end{equation}
% This is already implemented, where as for DFTB3, the DFTB basis function is also dependent on the charge, and the integral derivative is thus different:
% \begin{equation}
% \frac{\mathrm{\partial} M_{ij}}{\mathrm{\partial}q_k} =
%     f(R_{ij})\frac{\mathrm{\partial}}{\mathrm{\partial}q_k}  \iint \frac{\phi_i^\mathrm{cpe}\left(\mathbf{r}, q_i\right)\phi_j^\mathrm{dftb3}\left(\mathbf{r'}, q_j\right)}{\left| \mathbf{r} - \mathbf{r'}\right|} \mathrm{d}\mathbf{r}\mathrm{d}\mathbf{r'}
% \end{equation}
% This is already implemented (also for DFTB3).
% 
% 
\end{document}
 
